\documentclass[12pt,a4paper]{article}
\usepackage[utf8]{inputenc}
\usepackage{geometry}
\usepackage{graphicx}
\usepackage{listings}
\usepackage{xcolor}

% Configuration de la page
\geometry{margin=1in}

% Configuration pour le code source
\lstset{
    language=Java,
    basicstyle=\ttfamily\small,
    keywordstyle=\color{blue},
    stringstyle=\color{red},
    commentstyle=\color{gray},
    breaklines=true,
    numbers=left,
    stepnumber=1,
    numberstyle=\tiny,
    backgroundcolor=\color{gray!10}
}

\title{Projet DAO/MVC en Java : Application de Gestion des Employés}
\author{Votre Nom}
\date{\today}

\begin{document}

\maketitle

\tableofcontents

\newpage

\section{Introduction}
Ce rapport présente une application développée en Java basée sur l'architecture **MVC** (Modèle-Vue-Contrôleur) avec un accès aux données via la couche **DAO** (Data Access Object).  
L'application permet la gestion des employés dans une entreprise en incluant différents rôles et postes.

\section{Objectifs}
\begin{itemize}
    \item Implémenter une architecture MVC robuste.
    \item Créer une application avec des rôles et des postes spécifiques.
    \item Utiliser des classes DAO pour séparer la logique d'accès aux données.
\end{itemize}

\section{Architecture du Projet}
L'architecture MVC est composée de :  
\begin{itemize}
    \item \textbf{Modèle :} Gère les données (DAO).
    \item \textbf{Vue :} Affiche les informations à l'utilisateur.
    \item \textbf{Contrôleur :} Gère les actions de l'utilisateur.
\end{itemize}

\section{Étapes de Création du Projet}

\subsection{Création du Modèle (DAO)}
Le modèle représente les objets métiers et l'accès aux données.

\subsubsection{Classe Employee}
\begin{lstlisting}package Model;



public class Employee {
    public enum Role {
        EMPLOYEE,
        ADMIN
    }

    public enum Poste {
        INGENIEUR_ETUDE_ET_DEVELOPPEMENT,
        TEAM_LEADER,
        PILOTE
    }
    private int id;
    private String nom;
    private String prenom;
    private String tel;
    private String email;



    private Double salaire ;
    private Poste poste ;

    public Role getRole() {
        return role;
    }

    public void setRole(Role role) {
        this.role = role;
    }

    private Role role ;

    public Employee(int id, String nom, String prenom, String tel, String email,double salaire , Role role , Poste poste) {
        this.salaire = salaire ;
        this.id = id;
        this.nom = nom;
        this.prenom = prenom;
        this.tel = tel;
        this.email = email;
        this.role = role ;
        this.poste = poste ;
    }
    public Double getSalaire() {
        return salaire;
    }
    public Poste getPoste() {
        return poste;
    }

    public void setPoste(Poste poste) {
        this.poste = poste;
    }
    public void setSalaire(Double salaire) {
        this.salaire = salaire;
    }
    public int getId() {
        return id;
    }

    public void setId(int id) {
        this.id = id;
    }

    public String getNom() {
        return nom;
    }

    public void setNom(String nom) {
        this.nom = nom;
    }

    public String getPrenom() {
        return prenom;
    }

    public void setPrenom(String prenom) {
        this.prenom = prenom;
    }

    public String getTel() {
        return tel;
    }

    public void setTel(String tel) {
        this.tel = tel;
    }

    public String getEmail() {
        return email;
    }

    public void setEmail(String email) {
        this.email = email;

    }

    @Override
    public String toString() {
        return String.format(
                "| %-5s | %-15s | %-15s | %-15s | %-25s | %-10s | %-20s | %-10s |",
                id, nom, prenom, tel, email, salaire, poste, role
        );
    }
}
package Model;

import DAO.EmployeeDAOImpl;
import java.util.List;

public class EmployeeModel {
    private final EmployeeDAOImpl dao;

    public EmployeeModel(EmployeeDAOImpl dao) {
        this.dao = dao;
    }

    public void addEmployee(Employee emp) {
        dao.add(emp);
    }

    public List<Employee> getAllEmployees() {
        return dao.findAll();
    }

    public void deleteEmployee(int id) {
        dao.delete(id);
    }

    public void updateEmployee(Employee emp, int id) {
        dao.update(emp, id);
    }

    public static int parseEmployeeId(String selectedEmployeeString) {
        try {
            return Integer.parseInt(selectedEmployeeString.split("\\|")[1].trim());
        } catch (Exception e) {
            throw new IllegalArgumentException("Invalid employee format.");
        }
    }
}

\end{lstlisting}

\subsubsection{Classe DAO}
\begin{lstlisting}
import java.util.ArrayList;
import java.util.List;

public class EmployeeDAO {
    private List<Employee> employees = new ArrayList<>();

    public void addEmployee(Employee e) {
        employees.add(e);
    }

    public List<Employee> getAllEmployees() {
        return employees;
    }
}
\end{lstlisting}

\subsection{Création du Contrôleur}
Le contrôleur gère la logique métier.

\begin{lstlisting}
package Controller;

import Model.Employee;
import Model.EmployeeModel;
import Vue.Vue;

import javax.swing.*;
import java.util.ArrayList;
import java.util.List;

public class EmployeeController {
    private final EmployeeModel model;
    private final Vue view;

    public EmployeeController(EmployeeModel model, Vue view) {
        this.model = model;
        this.view = view;

        initializeListeners();
    }

    private void initializeListeners() {
        view.getAjouter().addActionListener(e -> {
            try {
                Employee emp = new Employee(
                        0,
                        view.getNom().getText(),
                        view.getPrenom().getText(),
                        view.getTel().getText(),
                        view.getEmail().getText(),
                        Double.parseDouble(view.getSal().getText()),
                        Employee.Role.valueOf((String) view.getRoleComboBox().getSelectedItem()),
                        Employee.Poste.valueOf((String) view.getPostesComboBox().getSelectedItem())
                );
                model.addEmployee(emp);
                JOptionPane.showMessageDialog(view, "Employee added successfully!", "Success", JOptionPane.INFORMATION_MESSAGE);
                view.getAfficher().doClick();
            } catch (NumberFormatException ex) {
                JOptionPane.showMessageDialog(view, "Invalid salary value!", "Error", JOptionPane.ERROR_MESSAGE);
            } catch (IllegalArgumentException ex) {
                JOptionPane.showMessageDialog(view, ex.getMessage(), "Error", JOptionPane.ERROR_MESSAGE);
            }
        });

        view.getAfficher().addActionListener(e -> {
            List<Employee> allEmployees = model.getAllEmployees();
            List<String> employeeStrings = new ArrayList<>();
            employeeStrings.add(String.format(
                    "| %-5s | %-15s | %-15s | %-15s | %-25s | %-10s | %-20s | %-10s |",
                    "ID", "Nom", "Prenom", "Tel", "Email", "Salaire", "Poste", "Role"
            ));
            for (Employee emp : allEmployees) {
                employeeStrings.add(emp.toString());
            }
            String[] employeeArray = employeeStrings.toArray(new String[0]);
            JList<String> updatedList = new JList<>(employeeArray);
            view.setEmployeeList(updatedList);
            JPanel p3 = view.getP3();
            p3.removeAll();
            p3.add(new JScrollPane(updatedList));
            p3.revalidate();
            p3.repaint();
        });

        view.getSupprimer().addActionListener(e -> {
            String selectedEmployeeString = view.getEmployeeList().getSelectedValue();
            if (selectedEmployeeString == null) {
                JOptionPane.showMessageDialog(view, "No employee selected!", "Error", JOptionPane.ERROR_MESSAGE);
                return;
            }

            try {
                int id = EmployeeModel.parseEmployeeId(selectedEmployeeString);
                int confirm = JOptionPane.showConfirmDialog(view, "Are you sure you want to delete this employee?", "Confirm", JOptionPane.YES_NO_OPTION);
                if (confirm == JOptionPane.YES_OPTION) {
                    model.deleteEmployee(id);
                    JOptionPane.showMessageDialog(view, "Employee deleted successfully!", "Success", JOptionPane.INFORMATION_MESSAGE);
                    view.getAfficher().doClick();
                }
            } catch (IllegalArgumentException ex) {
                JOptionPane.showMessageDialog(view, ex.getMessage(), "Error", JOptionPane.ERROR_MESSAGE);
            }
        });

        view.getModifier().addActionListener(e -> {
            String selectedEmployeeString = view.getEmployeeList().getSelectedValue();
            if (selectedEmployeeString == null) {
                JOptionPane.showMessageDialog(view, "No employee selected!", "Error", JOptionPane.ERROR_MESSAGE);
                return;
            }

            try {
                int id = EmployeeModel.parseEmployeeId(selectedEmployeeString);
                Employee emp = new Employee(
                        id,
                        view.getNom().getText(),
                        view.getPrenom().getText(),
                        view.getTel().getText(),
                        view.getEmail().getText(),
                        Double.parseDouble(view.getSal().getText()),
                        Employee.Role.valueOf((String) view.getRoleComboBox().getSelectedItem()),
                        Employee.Poste.valueOf((String) view.getPostesComboBox().getSelectedItem())
                );
                model.updateEmployee(emp, id);
                JOptionPane.showMessageDialog(view, "Employee updated successfully!", "Success", JOptionPane.INFORMATION_MESSAGE);
                view.getAfficher().doClick();
            } catch (NumberFormatException ex) {
                JOptionPane.showMessageDialog(view, "Invalid salary value!", "Error", JOptionPane.ERROR_MESSAGE);
            } catch (IllegalArgumentException ex) {
                JOptionPane.showMessageDialog(view, ex.getMessage(), "Error", JOptionPane.ERROR_MESSAGE);
            }
        });
    }
}

\end{lstlisting}

\subsection{Création de la Vue}
La vue affiche les résultats.

\begin{lstlisting}
package Vue;
import DAO.*;
import Model.Employee;

import javax.print.DocFlavor;
import javax.swing.*;
import javax.swing.border.Border;
import java.awt.*;
import java.util.ArrayList;
import java.util.List;

public class Vue extends JFrame {

    private JPanel p1;
    private JPanel p2;
    private JPanel p3;
    public JPanel getP3(){
        return p3 ;
    }

    private JPanel p4;
    private JComboBox<String> postesComboBox ;
    private JComboBox<String> roleComboBox ;
    private JButton ajouter ;
    private JButton modifier;
    private JButton supprimer ;
    private JButton afficher ;


    private JTextField tel;
    private JTextField sal  ;
    private JTextField nom ;
    private JTextField prenom ;
    private JTextField email  ;
    private  JList<String> employeeList ;
    public  JList<String> getEmployeeList(){
        return employeeList ;
    };
    public  void setEmployeeList(JList<String> employeeList){
        this.employeeList =employeeList   ;
    } ;
    public Vue() {
        setTitle("App");
        setSize(1920, 1080);
        setDefaultCloseOperation(JFrame.EXIT_ON_CLOSE);
        EmployeeDAOImpl eImp = new EmployeeDAOImpl();
        setDefaultCloseOperation(JFrame.EXIT_ON_CLOSE);
        p1 = new JPanel();
        p2 = new JPanel();
        p3 = new JPanel();
        p4 = new JPanel();
        p1.setLayout(new BorderLayout());
        p2.setLayout(new GridLayout(7,2,10,10));

        p3.setLayout(new GridLayout());
        p4.setLayout(new GridLayout());
        add(p1);
        p1.add(p2 , BorderLayout.NORTH);
        p1.add(p3 , BorderLayout.CENTER);
        p1.add(p4 , BorderLayout.SOUTH);


        p2.add(new JLabel("Nom:"));
        nom =  new JTextField();p2.add(nom);
        p2.add(new JLabel("Prenom:"));
        prenom =  new JTextField();p2.add(prenom);
        p2.add(new JLabel("Email:"));
        email =  new JTextField();p2.add(email);
        p2.add(new JLabel("Telephone:"));
        tel = new JTextField();p2.add(tel);
        p2.add(new JLabel("Salaire:"));
        sal =  new JTextField();p2.add(sal);

        p2.add(new JLabel("Role:"));
        List<Employee.Role> roles = eImp.findAllRoles() ;
        String[] roleStrings = roles.stream()
                .map(Enum::name)
                .toArray(String[]::new);
        roleComboBox = new JComboBox<String>(roleStrings);
        p2.add(roleComboBox);

        p2.add(new JLabel("Poste:"));
        List<Employee.Poste> postes = eImp.findAllPostes() ;
        String[] postesStrings = postes.stream()
                .map(Enum::name)
                .toArray(String[]::new);
        postesComboBox = new JComboBox<String>(postesStrings);
        p2.add(postesComboBox);



        //P3 Container
        List<Employee> all_e = eImp.findAll();
        List<String> allString = new ArrayList<String>();
        allString.add(String.format(
                "| %-5s | %-15s | %-15s | %-15s | %-25s | %-10s | %-20s | %-10s |",
                "ID", "Nom", "Prenom", "Tel", "Email", "Salaire", "Poste", "Role"
        ));
        for (Employee e : all_e){
            allString.add(e.toString());
        }
        String[] allStringArray = allString.toArray(new String[0]);

        employeeList = new JList<String>(allStringArray);
        p3.add(employeeList);


        //p4 Container

        p4.setLayout(new FlowLayout());
        this.ajouter = new JButton("Ajouter");
        this.modifier = new JButton("Modifier");
        this.supprimer = new JButton("Supprimer");
        this.afficher = new JButton("Afficher");
        p4.add(this.ajouter);p4.add(this.modifier);p4.add(this.supprimer);p4.add(this.afficher);
        setVisible(true);
    }

    public JComboBox<String> getPostesComboBox() {
        return postesComboBox;
    }

    public JComboBox<String> getRoleComboBox() {
        return roleComboBox;
    }

    public JButton getAjouter() {
        return ajouter;
    }

    public JButton getModifier() {
        return modifier;
    }

    public JButton getSupprimer() {
        return supprimer;
    }

    public JButton getAfficher() {
        return afficher;
    }

    public JTextField getTel() {
        return tel;
    }

    public JTextField getSal() {
        return sal;
    }

    public JTextField getNom() {
        return nom;
    }

    public JTextField getPrenom() {
        return prenom;
    }

    public JTextField getEmail() {
        return email;
    }

    public JPanel getP1() {
        return p1;
    }

}

\end{lstlisting}

\subsection{Création de la Classe Main}
\begin{lstlisting}
import DAO.EmployeeDAOImpl;
import Model.EmployeeModel;
import Vue.Vue;
import Controller.EmployeeController;

public class Main {
    public static void main(String[] args) {
        EmployeeDAOImpl dao = new EmployeeDAOImpl();
        EmployeeModel model = new EmployeeModel(dao);
        Vue view = new Vue();

        // Pass the model and view to the controller
        new EmployeeController(model, view);
    }
}

\end{lstlisting}

\section{Conclusion}
Ce projet met en œuvre l'architecture MVC et l'utilisation des DAO pour une application simple de gestion des employés.  
Chaque couche est bien séparée, permettant une évolutivité et une maintenance optimales.

\end{document}
